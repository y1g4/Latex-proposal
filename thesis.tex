\documentclass[12pt]{thesis} %%%%%%%%%%%%%%%%%%%%%%%%%%%%%%%%%%%%%%%%%%%%%%%%%%

%%% preample %%%%%%%%%%%%%%%%%%%%%%%%%%%%%%%%%%%%%%%%%%%%%%%%%%%%%%%%%%%%%%%%%%


%%% packages %%%%%%%%%%%%%%%%%%%%%%%%%%%%%%%%%%%%%%%%%%%%%%%%%%%%%%%%%%%%%%%%%%

%%%%%%%%%%%%%  %%%%%%%%%%%%%%%%%%%%%%%%%%%%%%%%

%\documentclass[12pt]{article}
%mmmmmmmmmmmmmmmmmmmmmmmm
\setcounter{tocdepth}{3}       %Depth of Levels in the table of Contents i.e. words after 4.2.4.1

\setcounter{secnumdepth}{3}    %Depth of Levels in the table of Contents i.e. 4.2.4.1 will appear
%mmmmmmmmmmmmmmmmmmmmmmmm


\usepackage{tabularx}

\usepackage{enumerate}
\usepackage{amsmath}
\usepackage{amsfonts}

%\usepackage{amssymb}
%\usepackage{graphicx}
\usepackage{times}
%\usepackage{multirow}			% for many rows in a cell of a table
%\usepackage{setspace}
\usepackage{float}				% for positioning figures, tables in exact positions i.e [H] option

%%%%%%%%%%%%%%%%%%%%%% End of Packages for Proposal %%%%%%%%%%%%%%%%
%\usepackage{url}



\usepackage{url}

%% Define a new 'leo' style for the package that will use a smaller font.
\makeatletter
\def\url@leostyle{%
  \@ifundefined{selectfont}{\def\UrlFont{\sf}}{\def\UrlFont{\small\ttfamily}}}
\makeatother
%% Now actually use the newly defined style.
\urlstyle{leo}




\usepackage{cite} %%%% this arranges multiple refs in order but removes the color and hyperlink %%%%%%%%%%%%%%%%%



\usepackage[T1]{fontenc}        % euro quality fonts [T1] (togeth. w/ textcomp)
\usepackage{textcomp, amssymb}  % additional symbols (there are more packages)
%%%%%%%%%\usepackage[latin1]{inputenc}   % umlaute in input file
\usepackage{setspace}           % doublespacing
\usepackage{anysize}            % margin package sets tighter margins
\usepackage[all]{xy}            % creating figures within latex
%\usepackage[tight]{subfigure}% subfigures: figures within figures

%\marginsize{1.2in}{0.9in}{1.1in}{0.9in} % small margins
\marginsize{1.2in}{0.9in}{0.5in}{1.5in} % small margins

\usepackage{ifpdf}              % if pdflatex then ... else ...
\ifpdf
  \pdfadjustspacing=1           % make pdflatex behave like latex
  \usepackage{aeguill}          % PS converted CM fonts for better acro preview
  \usepackage[pdftex]{graphicx} % graphics packages
  \usepackage[pdftex]{color}    % color packages
  \usepackage[pdftex]{thumbpdf} % create thumbnails (run thumbpdf as well)
  \usepackage[pdftex,colorlinks,%
              pagebackref=true, % bibliography -> text
              linktocpage=true, % toc etc: make page number active (not name)
              plainpages=false, % distinguish roman and arabic pagenumbers
              bookmarksopen=true,%
              bookmarksnumbered=true,%
              pdfauthor={yiga frank},%
              pdftitle={a mobile  application  to  exploit market    for    farmers},%
              pdfsubject={Undergraduate Research},%
              pdfkeywords={exploit},%
             ]{hyperref}        % clickabe references
\else
  \usepackage[hypertex,
              plainpages=false, % distinguish roman and arabic pagenumbers
              linktocpage=true, % toc etc: make page number active (not name)
             ]{hyperref}        % clickabe references in .dvi
                                % purposely included before color package
  \usepackage[dvips]{color}     % color packages; needed by xy
  \usepackage[dvips]{graphicx}  % graphics packages
  \graphicspath{ {./images/} }
\fi


% hyperref must be the second last package and glossary the last package

% index

\usepackage{makeidx}                       % for \printindex
\makeindex                                 % creates paper.idx index file

% glossary
%\usepackage[style=super, cols=3]{glossary} % for \printclossary
%\makeglossary                              % creates paper.glo glossary file
\usepackage{listings,multicol}
 \usepackage{courier}
 %\lstlistoflistings
\definecolor{darkgray}{rgb}{0.95,0.95,0.95}
\lstset{language=XML}
%\lstset{backgroundcolor=\color{darkgray}}
%\lstset{numbers=left, numberstyle=\tiny, numbersep=5pt}
\lstset{keywordstyle=\color{red}\bfseries\emph}
%\lstset{fonttype = \fontfamily{courier}\selectfont

\renewcommand{\topfraction}{.9}
\renewcommand{\bottomfraction}{.9}
\renewcommand{\textfraction}{.1}
\renewcommand{\floatpagefraction}{.9}
\renewcommand{\dbltopfraction}{.9}
\renewcommand{\dblfloatpagefraction}{.9}
\setcounter{topnumber}{50}
\setcounter{bottomnumber}{50}
\setcounter{totalnumber}{50}
\setcounter{dbltopnumber}{50}
\DeclareGraphicsExtensions{.pdf,.png,.jpg}
\definecolor{maroon}{rgb}{0.5,0,0}
\definecolor{darkgreen}{rgb}{0,0.5,0}

\lstdefinelanguage{XML}
{
  basicstyle=\ttfamily,
  morestring=[s]{"}{"},
  morecomment=[s]{?}{?},
  morecomment=[s]{!--}{--},
  commentstyle=\color{darkgreen},
  moredelim=[s][\color{black}]{>}{<},
  moredelim=[s][\color{red}]{\ }{=},
  stringstyle=\color{blue},
  identifierstyle=\color{maroon}
}

%%% style and finetuning %%%%%%%%%%%%%%%%%%%%%%%%%%%%%%%%%%%%%%%%%%%%%%%%%%%%%%

\pagestyle{plain}               % pagestyle: headings, empty, plain

% new theorems
\newtheorem{example}{Example}
\newtheorem{proof}{Proof}




%%%%%%%%%%%%%%%%%%%%%%%%%%%%%%%%%%%%%%%%%%%%%%%%%%%%%%%%%%%%%%%%%%%%%%%%%%%%%%%%%%%%%%%%%%
% this is for including ticks(checks) in a table
\usepackage{tikz}
\usepackage{calc}
\def\checkmark{\tikz\fill[scale=0.4](0,.35) -- (.25,0) -- (1,.7) -- (.25,.15) -- cycle;}
\def\scalecheck{\resizebox{\widthof{\checkmark}*\ratio{\widthof{x}}{\widthof{\normalsize x}}}{!}{\checkmark}}

\usepackage{booktabs}
\usepackage{multirow}
\usepackage{adjustbox}
\usepackage{hyperref}
\usepackage{bookmark}
\usepackage{subcaption}
%\usepackage{lipsum}
%\usepackage{subfig}
%\usepackage{graphicx,lipsum,afterpage,subcaption}
%\usepackage{enumitem}
% The booktabs package is needed to typeset good looking tables. I’ve written a post on the booktabs % package earlier. The multirow package is needed to create cells spanning multiple rows in your
% that's defined it - now for a test

%%%%%%%%%%%%%%%%%%%%%%%%%%%%%%%%%%%%%%%%%%%%%%%%%%%%%%%%%%%%%%%%%%%%%%%%%%%%%%%%%%%%%%%%%%


%%% document %%%%%%%%%%%%%%%%%%%%%%%%%%%%%%%%%%%%%%%%%%%%%%%%%%%%%%%%%%%%%%%%%%

\begin{document}


\ifpdf\pdfbookmark[1]{Title}{label:title}\fi              % do not use titlepage environment, because it does not increase page counter.

\thispagestyle{empty}%%%%%%%%%%%%%% forces no page number on the page %%%%%%%


\begin{center}
\setstretch{1.667}
{\bf\LARGE [Research Title/Topic]}
\par\vspace{3mm}
\begin{figure}[H]
    \centering
    \includegraphics[width=.15\linewidth]{images/MUST-logo.png}
    \label{fig:2019cases}
\end{figure}
\par\vspace{2mm}

\par By

\par\vspace{4mm}

\par{\bf [Full Name]}

[Registration Number]\\

[Degree Program/Course]\\

[Email address], [Phone number]

\par\vspace{5mm}

Supervisor\\

\par{\bf [Supervisor's Name}

Department of [Department Name]\\

[Supervisor's Email], [Supervisor's Phone Number]

\par\vspace{10mm}
%
A Project Proposal Submitted to the Faculty of Computing and Informatics for
the Study Leading to a Project in Partial Fulfillment of the requirements for the
Award of the Degree of Bachelor of computer science of Mbarara University of
Science and Technology

\par\vspace{3mm}

\par\vspace{16mm}
\par {\bf [[Month, Year]}

\end{center}



%%%%%%%%%%%%%%%%%%%%%%%%%%%%%%%%%%%%%%%%%%%%%%%%%%%%%%%%%%%%%%%%%%%%%%%%%%%%%%%


%%\newpage                                \vspace*{4cm}

\begin{flushright}
\begin{minipage}[]{90mm}
In the hallway of the university Wittgenstein asked a colleague: ``I've always wondered why for so long people thought that the sun revolved around the earth.''\\

``Why?'' said his surprised interlocutor, ``well, I suppose it just looks that way.''\\

``Hmm'', retorted Wittgenstein, ``and what would it look like if the earth revolved around the sun?''\\

This puzzled the interlocutor.
\end{minipage}
\end{flushright}


%%%%%%%%%%%%%%% set roman numerals for page numbers %%%%%%%%%%%%%%%%%%%%%
\newpage\pagenumbering{roman}

%\newpage\input{declaration.tex} %%%%%% call this file %%%%%%%%%
\newpage\chapter*{Salutation}


\vspace{1mm}

%\par\vspace{8mm}
\par\vspace{1mm}
This project proposal has been submitted with the approval of the following
supervisor.


%%%%%%%%%%%%\vfill %%%%%%%%%%%% creates huge space between paragraphs %%%%%%%%%%%% automatically

\vspace{5mm}

%%
%\parbox[t]{50mm}{
Signature:  \rule{50mm}{0.5pt}\\
\par\vspace{8mm}
\textbf{Supervisor's Name}\\
Supervisor
%%
%%
\par\vspace{10mm}
Date: \rule{50mm}{0.5pt}\\
%}\hfill
%\parbox[t]{50mm}{
%\rule{50mm}{0.5pt}\\Supervisor \\ Professor Jude Lubega
%%
%\par\vspace{10mm}
%\rule{50mm}{0.5pt}\\Date
%%
%%\par\vspace{10mm}
%%\rule{50mm}{0.5pt}\\Thesis Reader Dr.~Name
%%
%%\par\vspace{10mm}
%%\rule{50mm}{0.5pt}\\Thesis Reader Dr.~Name
%}
 %%%%%% call this file %%%%%%%%%

{\doublespacing
\newpage\ifpdf\pdfbookmark[1]{Acronyms}{label:ack}\fi\chapter*{List of acronyms}
\vspace{5mm}




%\bigskip\medskip

% ---------------------------------------------------------------------- 
\newpage\ifpdf\pdfbookmark[1]{List of Figures}{label:lof}\fi     \listoffigures
%\newpage\ifpdf\pdfbookmark[1]{Dedication}{label:ded}\fi\chapter*{Dedication}
\begin{center}
\
\
\
\
\
\
\
\
\par\vspace{20mm}
\
\
\
\
\
\

[Your Dedication]

\end{center}

%\bigskip\medskip

% ---------------------------------------------------------------------- 
\newpage\ifpdf\pdfbookmark[1]{Abstract}{label:abst}\fi\chapter*{Abstract}

\vspace{1mm}

%\par\vspace{8mm}
\par\vspace{1mm}
[Your Abstract]





}

\newpage\ifpdf\pdfbookmark[1]{Table of Contents}{label:toc}\fi \tableofcontents

%\newpage\ifpdf\pdfbookmark[1]{List of Tables}{label:lot}\fi       \listoftables


{\doublespacing
\newpage\pagenumbering{arabic} %%%% force the rest to have arabic pages %%%%%%%%

\newpage\chapter{Introduction}
\label{chapter:introduction} %%%%%%%%%%%%%%%%%%%%%%%%%%%%
%\section{Introduction}
\section{Introduction}
\label{sec:intro}


\section{Background}
\label{sec:background}


\section{Problem Statement} \label{problem}


\section{Objectives} \label{researchobjectives}
\subsection{Main Objective }
\noindent The main objective is...

\subsection{Specific Objectives}
\begin{enumerate}[(i)]
\item To review...
\item To develop...
\item To test...
\end{enumerate}
%\newpage
\section{Research Questions} \label{researchquestions}
\begin{enumerate}[(i)]
\item What are...
\item What are...
\item How best...
\end{enumerate}
\section{Project Significance} \label{researchsignificance}
\subsection{Value Proposition}

\subsection{Innovation}

\subsection{Impact}

\subsection{Business Component}

%\newpage
\section{Scope of the Study}
\noindent The study will be carried out...
%\newpage

%%%%%%%%%%%%%%%%%%%%%%%%%%%%% Literature Review Section%%%%%%%%%%%%%%%%%%%%%%%%%%%%%%%%%%%%%%
\chapter{Literature Review}\label{chapter:Literature_Review}
\section{Literature Review}
In this section, we present the findings from several literal sources with regard to the major concepts associated with this research. Information obtained from several journal articles, conference papers, relevant workshops, reputable websites, and published books are compiled to provide theoretical grounding and practical justification for the problem under investigation.

\section{...}


\section{...}

\section{...}

\chapter{Methodology}\label{chapter:Methodology}
\section{Methodology Name/Overview}



% Bibliography and Glossary          (\phantomsection is needed for hyperlinks)

\newpage\phantomsection%
\addcontentsline{toc}{chapter}{\bibname}              % add Bibliography to TOC

%%%%%%%%%%%%%%%%%%%%%%%%%%%%%%%%%%%%%%%%%%%%%%%%%%%%%%%%%%%%%%%%%%%%%

%%%%%%%%%%%%%%\bibliographystyle{alpha}\bibliography{references}

%%%%%%%%%%%%%%%%%%%%%%%%%%%%%%%%%%%%%%%%%%%%%%%%%%%%%%%%%%%%%%%%%%%%%%

%%%%%%%%%%%%%%%%%%%%%% nabaasa %%%%%%%%%%%%%%%%%%%%%%%%%%%%%%%%%%%%%

\bibliographystyle{plain} % or {acm} % but {IEEEannot} does not work
%\bibliographystyle{unsrturl}
\bibliography{references}
%%%%%%%%%%%%%%%%%%%%%% end nabaasa %%%%%%%%%%%%%%%%%%%%%%%%%%%%%%%%%%%

\newpage\appendix
\chapter{Appendices}
\vspace{5mm}


%\bigskip\medskip

% ---------------------------------------------------------------------- 


\newpage\phantomsection%
\addcontentsline{toc}{chapter}{\indexname}                   % add Index to TOC
%\printindex                        %% prints index page at the end %%%%

\newpage\phantomsection%
\addcontentsline{toc}{chapter}{Glossary}                  % add Glossary to TOC
%\printglossary                     %%prints glossary page at the end %%%

\end{document}

%%%%%%%%%%%%%%%%%%%%%%%%%%%%%%%%%%%%%%%%%%%%%%%%%%%%%%%%%%%%%%%%%%%%%%%%%%%%%%% 